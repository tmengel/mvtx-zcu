\documentclass{scrartcl}


\usepackage{gitdags}
\usepackage{subfigure}

\makeatletter
\def\verbatim{\footnotesize\@verbatim \frenchspacing\@vobeyspaces \@xverbatim}
\makeatother

\title{Git restructuring}

\begin{document}
\maketitle

\abstract{This report is written to describe the restructuring of the
  WP10 git repository, the motivation behind it, implementation
  details, and the adapted procedures needed to work with the
  introduced concept of subtrees. Any comments, extensions or
  feedback on the procedures is welcome.}

\tableofcontents

\section{Motivation}

With progressing developments the existing structure of the git
repository shows to be restrictive in terms of extending. This is due
to several factors
\begin{itemize}
\item The emerging of multiple projects (Readout chain, CRU emulator,
  Radiation testing)
\item The need to target multiple platforms (RUv0, RUv1)
\item A common code base to different projects
\item A shared, code base between different projects, with the need of
  modifications and specialisations for implementations
\end{itemize}

To avoid manual copies of sources and the resulting problems of
merging changes, drifting versions and manual keeping track of
modifications and bugfixes for all projects, we decided to refine the
structure of the project repositories. The new structure should be
able to cope with multiple projects, multiple versions and multiple
platforms while keeping the shared sources common to each other.

\subsection{List of Requirements}
\begin{itemize}
\item Maintaining multiple projects
\begin{itemize}
\item Mix of shared and project specific sources
\item Different target platforms (FPGA, board)
\end{itemize}
\item Maintaining shared sources
  \begin{itemize}
  \item Synchronisation mechanism between projects
  \item Possibility to keep different versions for different projects,
    while still sharing the same base (e.g for introducing bug-fixes
    and features efficiently)
  \end{itemize}
\item Code sharing between projects
\begin{itemize}
\item Share code parts with projects outside WP10 (Probecard)
\end{itemize}
\end{itemize}

\subsection{Outline}
The rest of the report is structured as follows: Section\ref{sec:structure} describes the concept and implementation of the
new structure. Section\ref{sec:working} describes different work
flows and use cases for working with the new structure.

\section{New Repository Structure}
\label{sec:structure}
A new structure of the repository was devised to meet the set
requirements. This is done via multiple changes.

\subsection{Concept}
\begin{enumerate}
\item The source code structure changes to be organized in modules
  \begin{itemize}
  \item A Module is a contained collection of related sources (RTL
    code, testbench code, model code, software, constraints)
  \item Examples for a module are the alpide frontend, the usb
    interface, the wishbone or the gbt implementation.
  \end{itemize}
\item Modules shared between projects will be moved to a separate git
  repository.
\item Each project moves to a separate git repository
  \begin{itemize}
  \item Project code is contains project specific sources, plus
    sources from a module
\item Using git methodology, a projects embeds module sources into
  it's repository (while keeping the link/history to the submodule
  repository)
\end{itemize}
\end{enumerate}

\subsection{File and Folder structure}
The base idea of the existing structure is kept, keeping the
separation between software an firmware sources. For firmware sourcse,
the separation of bench, rtl and constraints is kept. Folders for
simulation and documentation stay the same. The main difference is the
introduction of a module folder. This folder contains the submodule
directiories, which have their own folder structure.

\subsubsection{Project structure example}
The following example is a canonical folder structure of a project.
\begin{verbatim}
. (Project top folder)
|-- README.md
|-- doc (documentation)
|-- filelist (List of project specific sources)
|-- modules (Contains all modules used by this project)
|   `-- DEPENDENCIES (Define dependencies of this project to its submodules)
|-- sim (Simulation scripts)
|-- source (source code)
|   |-- rtl
|   |-- bench
|   |-- constraints
|   `-- ip
|-- verification (formal verification scripts)
|-- vivado (Vivado scripts)
`-- software (software files, e.g scripts and tools)
\end{verbatim}

\subsubsection{Module structure example}
The following example is a canonical folder structure for a module.

\begin{verbatim}
. (Module top folder)
|-- README.md
|-- doc (documentation)
|-- filelist (List of project specific sources)
|-- sim (Simulation scripts)
|-- source (source code)
|   |-- rtl
|   |-- bench
|   |-- constraints
|   `-- ip
`-- software (module related software files)
\end{verbatim}

\subsection{Module dependency file}
To define which modules are required by a project, a module dependency
file has to be created. This is used for later generating the full
filelist of the project, and to define the order of the modules. If
module B depends on module A, module A needs to be listed before in
the dependency file. The dependency file must be placed in the modules
folder with the name \verb|DEPENDENCIES|. The syntax used is a JSON
key value pair, with the module name as key, and the location as
value. The following example shows the module \verb|DEPENDENCY| file
of the \verb|RUv0_RDO| project:

\begin{verbatim}
{
    "common":"modules/common",
    "wishbone":"modules/wishbone",
    "alpide_frontend": "modules/alpide_frontend",
    "voltage_control": "modules/voltage_control",
    "gbt" : "modules/gbt",
    "i2c" : "modules/i2c",
    "usb_if" : "modules/usb_if"
}
\end{verbatim}

\subsection{File lists}
The collection of files needed for using a module (both simulation and
synthesis) are decentralized and bound to the module itself. Each
module has a folder filelist, which contains filelists for defining
what sources are needed to use the modules. In addition, each project
has a filelist folder, which provides the additional files provided by
the project itself.

The filelist folder contains multiple lists, separated by file type
and their use (synthesis,simulation).

\begin{description}
\item[sdc\_sources.txt] List of constraints
\item[vhdl\_sources.txt] List of VHDL sources
\item [vlog\_sources.txt] List of (System)Verilog sources
  \item[ip\_sources.txt] List of Vivado IP sources
  \item [vhdl\_sources\_sim.txt] List of VHDL sources used in Simulation
    (but not in synthesis)
\item [vlog\_sources\_sim.txt] List of (System)Verilog soures used in Simulation
  (but not in synthesis)
\end{description}

Each filelist lists one source file per line, using their relative
path (starting at the top level of the module). For example, the path
to a file \verb|test.vhd| within the rtl folder of a module would be
\verb|source/rtl/test.vhd|. Special cases are given for the following
source lists:
\begin{description}
\item[vlog\_sources(\_sim).txt] Include directiories can be added
  using the usual \verb|+incdir+| structure, but include resolution
  works differently. To link to a directory, the name of the module
  has to be prepended, in the form of
  \verb+$<modulename>_(bench|rtl|ip)+. For example, including the
  folder \verb|common_includes| within the common module, inside a
  vlog sourcefile results in the following syntax:
\begin{verbatim}
+incdir+$common_rtl/common_includes <sourcefile>
\end{verbatim}
\item [ip\_sources.txt] For adding IP sources, the \verb|.xci(x)| file of
  the ip core has to be referenced.
\end{description}

\subsection{Scripts and Tools}

With separation into modules and the introduction of modules and their
filelists, the handling of synthesis and simulation scripts was
changed. Instead of manually maintaining a list of sources needed for
the different flows, a collective filelist is generated in the top
folder of the project based on the filelists of the project and each
module. These filelists are used by the script for creating the vivado
project as well as simulation scripts. This requires the user
re-generate the filelist if a structural change is introduced. This is
handled by a \verb|create_filelist.py| script. The recommended way of
using this process is to work with the provided Makefiles. These
handle the re-creation of the filelists automatically before creating
a project or running simulations. If a manual creation is needed, the
script can be called from a terminal. It is located in the software
folder of the common module
(\verb|software/py/create_filelist.py|). When calling, it needs the
absolute path to the project top level repository as argument.

\subsection{Git refactoring process and methodology}

The central aspect of the new structure is the splitting of the code
into multiple repositories: One repository for each project, and one
repository for each submodule. The central element of this structure
is to embed repositories within each other while keeping track of the
modifications, history and the link between repositories.

The concept used to implement this structure is the git subtree flow.
With git subtrees, the embedded repository gets copied into the parent
container repository, as if it would be manually moved
there. Internally, git still keeps track of the remote repository
used, it's versions and the history between commits. From a project
point of view, however, this mechanism is hidden from the user. The
project can be used without having knowledge of the existence of the
embedded subtrees. Only the person in charge of maintaining a subtree
needs to know the inner workings and handle the synchronisation
process between versions of the subtree.

The disadvantage of this workflow is an increase of complexity in the
process of maintaining and updating modules. This will increase the
time required to introduce changes to a module, as each modification
has to be extracted to be moved to a submodule, merged in the
submodule and then distributed to all other projects. However, as this
is needed by a limited number of users (mainly the maintainers of the
project and submodules), the majority of users can work without this
knowledge.

\section{Working with the new structure}
\label{sec:working}

The restructuring of the repositories introduces new workflows and
conventions to be considered when developing and using a project. This
section is a collection of common workflows and tasks required to work
with a project or submodules.

\subsection{Working with a container project repository}

Each project is located in a separate git repository. To work with a
project, it has to be downloaded, or cloned to a local
folder. Compilation and simulation are provided with the project. The
following examples will assume to work on the Readout project of RUv0.

\subsubsection{Download, Compiling, simulating}
The first step is to download the project.

\begin{verbatim}
$ git clone ssh://git@gitlab.cern.ch:7999/alice-its-wp10-firmware/RUv0_RDO.git
$ cd RUv0_RDO
\end{verbatim}

The folder structure of the checked out project is as follows:
\begin{verbatim}
.
|-- doc
|-- filelist
|   |-- sdc_sources.txt
|   |-- vhdl_sources.txt
|   |-- vlog_includes.txt
|   `-- vlog_sources_sim.txt
|-- Makefile
|-- modules
|   |-- alpide_frontend
|   |-- alpide_lightweight
|   |-- board_support_software
|   |-- common
|   |-- common_tmr
|   |-- DEPENDENCIES
|   |-- gbt
|   |-- gbt_fpga
|   |-- i2c
|   |-- usb_if
|   |-- voltage_control
|   `-- wishbone
|-- README.md
|-- sim
|   |-- libs
|   |-- libs_ius
|   |-- Readme.md
|   |-- run
|   |-- scripts
|   `-- src
|-- software
|   `-- py
|-- source
|   |-- bench
|   |-- constraints
|   |-- ip
|   `-- rtl
|-- verification
|   `-- formal
`-- vivado
    |-- ila_fsm
    `-- scripts
\end{verbatim}

A provided Makefile is used to build the project and perform
additional steps needed for synthesis or simulation. The first step is
to check the Paths defined in the Makefile and verify that they are
correct for the given system. The following variables need to be
checked:

\begin{description}
\item [VIVADO\_PATH] The path to the Vivado installation
\item [MODELSIM] The path to the modelsim installation
\end{description}

The following targets are provided (a full list of targets is
described in the Makefile itself).

\begin{description}
\item[make gui [s/i/b]] Creates a Vivado project and opens the
  Graphical Vivado program. If the optional parameter (s)ynthesis,
  (i)mplementation, or (bitstream) is provided, the script runs up
  past the indicated step.
\item[make\ compile\_simlib\_ius/compile\_simlib\_modelsim] Create the
  Vivado simulation libraries required by Cadence IUS simulator or modelsim
  simulator.
\end{description}

For simulation, environments for modelsim and Cadence IUS are
provided. These are located in the \verb|sim| folder. Cadence IUS
scripts are located in \verb|sim/run|, scripts for Modelsim are
provided in \verb|sim/scripts|. The Makefiles and tcl scrpits need to
be checked for proper settings of paths.
\begin{itemize}
\item Cadence IUS:
\begin{verbatim}
$ cd sim/run
$ make compile_pALPIDE4
$ make
\end{verbatim}
\item Modelsim:
\begin{verbatim}
$ cd sim/scripts
$ make alpide_lightweight
$ make
\end{verbatim}
\end{itemize}

The testbench in use is located in
\verb|source/bench/RUv0a_top_tb.sv|. By default, it uses an
interactive testbench, which is stimulated by a python script. This
script is located in \verb|software/py/RUv0_regression.py|. When
running the python script, it tries to communicate with the simulator
and exchange commands via virtual usb interface. After starting the
script, the simulation can be run. During execution, the simulation
can be paused and continued at any time.

Known issue: Due to a race condition, starting of the python script
might fail the first time as it is missing files created by the
simulation. When this happens, the simulation has to be started
first. The simulation will then fail (because it cannot connect to the
python script). After failing, the simulation can be reset, and the
python script started normally.

\subsubsection{Making modifications}

Modifications can be made as with any normal repository. Project
related files are located in the project source folder. New file
additions need to be added to the local container
filelists. Modifications within modules can be introduced without
problems. However, there are a few recommendations for introducing
changes in modules, which make it easier to merge to the submodule
repository by the maintainer.

\begin{itemize}
\item Changes which should be back ported to the submodule should be
  added in a separate commit, with descriptive commit message. The
  commit message should start with an indication, e.g
  \verb|[To submodule_X] <Msg>|.
\item The maintainer of the submodule should be notified of changes
  made to the submodule. The modifications should be stable and
  tested.
\item Modifications within the submodule, which should not be
  back ported, should be kept in separate commits, and indicated with a
  special message, e.g \verb|[DO NOT BACKPORT]|. This simplifies the
  work for the maintainer.
\end{itemize}

\subsection{Creating a new project}

For a new project, a new repository needs to be created. Depending on
the novelty and similarity to existing projects, this can be either
done from scratch, or as a clone of an existing project. The new
project needs to adhere to the existing folder and module structure,
define its module \verb|DEPENDENCY| file, and a Makefile for providing
the filelist creation.

\subsubsection{Adding an existing submodule to the project}

Adding a new module to a project requires it to be added to the module
\verb|DEPENDENCY| file, and to physically add the sources to the
project. For the \verb|DEPENDENCY| file, an additional line, with the
name and path to the module is added. Existing lines can be copied and
modified.

Adding the sources of the submodule to the project requires the use of
the \verb|git subtree| command. The commands to run are executed from
the project root.

\begin{enumerate}
\item First, the submodule repository is added as a remote
repository:

\begin{verbatim}
$ git remote add submodule_X \
$ ssh://git@gitlab.cern.ch:7999/alice-its-wp10-firmware/submodule_X.git
\end{verbatim}

\item Then, the module sources are injected using \verb|git subtree add|

\begin{verbatim}
$ git subtree add -m \
$ "submodule injection: inject submodule X into local repository, squash" \
$ --squash --prefix=modules/X submodule_X master
\end{verbatim}

  The command defines a commit message, the target folder of the
  module, and the branch of the module repository to use
\end{enumerate}

This generates a commit to the local repository, which adds all
sources of the module repository to the local project, keeping a link
to the remote.

\subsubsection{Updating a project with a new version of a submodule}

To update the local version of a module to a newer, updated version,
the \verb|git subtree pull| command is used. This is usually performed
by the maintainer of the submodule in cooperation with the maintainer
of a project (to prevent conflicts or broken code). Updating a
submodule requires the remote of the submodule to be added to the
local repository, in the same way as for adding a repository. After
this is done, the action for updating is a single command:

\begin{verbatim}
$ git subtree pull -m "update submodule X" \
$ --prefix=firmware/RUv0a/modules/X --squash submodule_X master
\end{verbatim}

A custom commit message, path to the module and a branch to pull from
can be provided. This performes a fetch and merge command, similar to
a normal pull. Local modifications will be merged with modifications
done in the remote repository. Merge conflicts will arise if local
modifications clash with remote modifications, and can be resolved
before commiting the merge.

\subsection{Maintaining a submodule repository}

Maintenance of a submodule involves the introductions of changes from
a project into the repository of the submodule, and the distribution of
changes, or new versions to other projects. The steps performed for
the maintainer of the module are the pushing from a project to the
module, maintaining the repository of the module (merges, introducing
changes), and distributing changes to the different projects.

\subsubsection{Filtering modifications}

A project might introduce changes to a module which should not be back
ported to the baseline of the submodule. These changes must be
filtered out before the feature branch is merged with the baseline
master branch of the module. Depending on the kind of these changes
and their granularity, different approaches can be taken to filter out
these changes. The smaller and more specific commits of a feature are,
the easier it is to filter unwanted modifications out. The following
list provides some suggestions on which git procedures to use to
achieve the desired result.

\begin{description}
\item [git cherry-pick] The cherry-pick command is used to apply
  changes introduced from a specific commit to the existing
  branch. The canonical workflow would be as follows:
  \begin{itemize}
  \item Create a temporary branch, which branches off before the first
    modification to the module was made
  \item Cherry pick all commits which should be included in the module
    update
  \item push the resulting branch to the submodule remote repository
  \item delete local branch, switch back to working branch
  \end{itemize}
\item [git rebase] When a bigger series of commits is to be kept, and
  only a few to be excluded, a rebase might be faster. With git
  rebase, modifications on a branch are reapplied. If this is done
  interactively, certain commits can be reshuffled or excluded. The
  canonical way of performing a rebase is as follows:
  \begin{itemize}
  \item Create a temporary branch starting at the last commit of the
    working branch.
  \item perform an interactive rebase onto the last commit before the
    module changes were performed
  \item pick all commits to be applied, remove all commits to be
    excluded. Only commits touching the module are relevant, others
    will be ignored when pushing later.
  \item Push the resulting branch to the submodule remote
  \item delete local branch, switch back to working branch
  \end{itemize}
\item [Corner cases] There might be corner cases or special cases
  which cannot be solved with cherry-pick or rebase. In these cases, a
  proper work-flow must still be defined or found out. Git usually
  provides the tools to rewrite the history as needed, e.g by
  splitting up commits, reshuffling commits, reverting commits,
  reapplying changes.
\item [Worst case, fix in submodule repository] The last resort is to
  revert modifications or patches in the submodule repository itself,
  either by reverting commits, excluding commits or simply manual
  reverting files. This can be done if necessary, but should be
  avoided if possible.
\end{description}


\subsubsection{Introducing changes to submodules made in a project
  repository}

Pushing changes to a module is a three-step process. First, the
modifications are pushed from the project repository to the module
repository. As with add/pull, this requires the submodule to be
registered as a remote. The command used for pushing to the module is
\verb|git subtree push|:

\begin{verbatim}
$ git subtree push -P firmware/RUv0a/modules/X submodule_X feature_branch
\end{verbatim}

This will create \verb|feature_branch| on the module repository, and
contains the modifications introduced within the project since the
last subtree pull.

The feature branch has to be checked, and when verified, merged to the
master of the submodule. As last step, the project should update the
submodule, to update the index to the merged version of the master.

At this point, it has to be decided if the change is beneficial to
other projects. If so, each project should update its submodule as
well.

\subsubsection{Creating a new submodule}

New submodules can be extracted from a project when necessary. For new
developments, the folder structure of modules should be used where
applicable. Any code part which can be seen as separate block, should
be created within the modules folder, and implemented as module (using
the correct structure and folder/files required). It is not necessary
to extract the folder immediately to a new repository, but keeping the
correct folder structure simplifies the process. As long as the module
doesn't need to be shared, it can be kept in the local project
repository.

\subsubsection{Moving a submodule to a new repository for sharing}

Sharing a module involves extracting it to a new repository. This is
done by using the \verb|git subtree split| command. The following
steps have to be performed:
\begin{enumerate}
\item Create a new repository for the module
\item Add the created repository as a remote to the local project
\item Perform a subtree split
\begin{verbatim}
$ git subtree split -P modules/Y -b split-Y
\end{verbatim}
  this splits module Y in modules to a local branch \verb|split-Y|
\item push the split of branch to the remote
\begin{verbatim}
$ git push -u submodule_Y split-Y:master
\end{verbatim}
  This pushes to remote \verb|submodule_Y| master
\item delete split-off branch, switch to working branch
\end{enumerate}

\subsection{State, comments, updates}

This document is not complete. Using this structure for longer time
will show which work flows need to be adapted and if common practices
can be extracted. Any helpful procedures are welcome to be added to
this document, as well as errors, comments, caveats or tips.

\appendix
\section{Selection git submodules and git subtrees}

Git provides two options to realise a structure of embedded
repositories from a technical point of view: \emph{Submodules} and
\emph{Subtrees}.

Git Submodules are a dedicated structure in git which embed a link to
another repository within the parent repository. The submodule acts as
a separate entity, with it's own branch history, commit procedure and
update procedure. The top project only stores a link to the hash of a
specific commit to the submodule as information.

Git Subtrees, on the other hand, are more a concept than a
feature. Here, the embedded repository gets copied into the parent
container repository, as if it would be manually moved
there. Internally, git still keeps track of the remote repository
used, it's versions and the history between commits. From a project
point of view, however, this mechanism is hidden from the user. The
project can be used without having knowledge of the existence of the
embedded subtrees. Only the person in charge of maintaining a subtree
needs to know the inner workings and handle the synchronisation
process between versions of the subtree.

After evaluating both versions it was decided to use the git subtree
concept, for the following reasons.

\begin{itemize}
\item No change in using git for people using the container projects,
  especially people interested in using the projects read-only.
\item A more centralised responsibility for a submodule, as it is the
  responsibility of a maintainer to update the code of a submodule.
\item A simpler way of introducing modifications on top of the
  submodule code, without the need of maintaining branches related to
  a project in a submodule repository. Modifications are still stored
  in the project repository.
\end{itemize}


\end{document}

%%% Local Variables:
%%% mode: latex
%%% TeX-master: t
%%% End:
